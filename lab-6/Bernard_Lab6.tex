% Options for packages loaded elsewhere
\PassOptionsToPackage{unicode}{hyperref}
\PassOptionsToPackage{hyphens}{url}
%
\documentclass[
]{article}
\usepackage{amsmath,amssymb}
\usepackage{lmodern}
\usepackage{iftex}
\ifPDFTeX
  \usepackage[T1]{fontenc}
  \usepackage[utf8]{inputenc}
  \usepackage{textcomp} % provide euro and other symbols
\else % if luatex or xetex
  \usepackage{unicode-math}
  \defaultfontfeatures{Scale=MatchLowercase}
  \defaultfontfeatures[\rmfamily]{Ligatures=TeX,Scale=1}
\fi
% Use upquote if available, for straight quotes in verbatim environments
\IfFileExists{upquote.sty}{\usepackage{upquote}}{}
\IfFileExists{microtype.sty}{% use microtype if available
  \usepackage[]{microtype}
  \UseMicrotypeSet[protrusion]{basicmath} % disable protrusion for tt fonts
}{}
\makeatletter
\@ifundefined{KOMAClassName}{% if non-KOMA class
  \IfFileExists{parskip.sty}{%
    \usepackage{parskip}
  }{% else
    \setlength{\parindent}{0pt}
    \setlength{\parskip}{6pt plus 2pt minus 1pt}}
}{% if KOMA class
  \KOMAoptions{parskip=half}}
\makeatother
\usepackage{xcolor}
\usepackage[margin=1in]{geometry}
\usepackage{color}
\usepackage{fancyvrb}
\newcommand{\VerbBar}{|}
\newcommand{\VERB}{\Verb[commandchars=\\\{\}]}
\DefineVerbatimEnvironment{Highlighting}{Verbatim}{commandchars=\\\{\}}
% Add ',fontsize=\small' for more characters per line
\usepackage{framed}
\definecolor{shadecolor}{RGB}{248,248,248}
\newenvironment{Shaded}{\begin{snugshade}}{\end{snugshade}}
\newcommand{\AlertTok}[1]{\textcolor[rgb]{0.94,0.16,0.16}{#1}}
\newcommand{\AnnotationTok}[1]{\textcolor[rgb]{0.56,0.35,0.01}{\textbf{\textit{#1}}}}
\newcommand{\AttributeTok}[1]{\textcolor[rgb]{0.77,0.63,0.00}{#1}}
\newcommand{\BaseNTok}[1]{\textcolor[rgb]{0.00,0.00,0.81}{#1}}
\newcommand{\BuiltInTok}[1]{#1}
\newcommand{\CharTok}[1]{\textcolor[rgb]{0.31,0.60,0.02}{#1}}
\newcommand{\CommentTok}[1]{\textcolor[rgb]{0.56,0.35,0.01}{\textit{#1}}}
\newcommand{\CommentVarTok}[1]{\textcolor[rgb]{0.56,0.35,0.01}{\textbf{\textit{#1}}}}
\newcommand{\ConstantTok}[1]{\textcolor[rgb]{0.00,0.00,0.00}{#1}}
\newcommand{\ControlFlowTok}[1]{\textcolor[rgb]{0.13,0.29,0.53}{\textbf{#1}}}
\newcommand{\DataTypeTok}[1]{\textcolor[rgb]{0.13,0.29,0.53}{#1}}
\newcommand{\DecValTok}[1]{\textcolor[rgb]{0.00,0.00,0.81}{#1}}
\newcommand{\DocumentationTok}[1]{\textcolor[rgb]{0.56,0.35,0.01}{\textbf{\textit{#1}}}}
\newcommand{\ErrorTok}[1]{\textcolor[rgb]{0.64,0.00,0.00}{\textbf{#1}}}
\newcommand{\ExtensionTok}[1]{#1}
\newcommand{\FloatTok}[1]{\textcolor[rgb]{0.00,0.00,0.81}{#1}}
\newcommand{\FunctionTok}[1]{\textcolor[rgb]{0.00,0.00,0.00}{#1}}
\newcommand{\ImportTok}[1]{#1}
\newcommand{\InformationTok}[1]{\textcolor[rgb]{0.56,0.35,0.01}{\textbf{\textit{#1}}}}
\newcommand{\KeywordTok}[1]{\textcolor[rgb]{0.13,0.29,0.53}{\textbf{#1}}}
\newcommand{\NormalTok}[1]{#1}
\newcommand{\OperatorTok}[1]{\textcolor[rgb]{0.81,0.36,0.00}{\textbf{#1}}}
\newcommand{\OtherTok}[1]{\textcolor[rgb]{0.56,0.35,0.01}{#1}}
\newcommand{\PreprocessorTok}[1]{\textcolor[rgb]{0.56,0.35,0.01}{\textit{#1}}}
\newcommand{\RegionMarkerTok}[1]{#1}
\newcommand{\SpecialCharTok}[1]{\textcolor[rgb]{0.00,0.00,0.00}{#1}}
\newcommand{\SpecialStringTok}[1]{\textcolor[rgb]{0.31,0.60,0.02}{#1}}
\newcommand{\StringTok}[1]{\textcolor[rgb]{0.31,0.60,0.02}{#1}}
\newcommand{\VariableTok}[1]{\textcolor[rgb]{0.00,0.00,0.00}{#1}}
\newcommand{\VerbatimStringTok}[1]{\textcolor[rgb]{0.31,0.60,0.02}{#1}}
\newcommand{\WarningTok}[1]{\textcolor[rgb]{0.56,0.35,0.01}{\textbf{\textit{#1}}}}
\usepackage{graphicx}
\makeatletter
\def\maxwidth{\ifdim\Gin@nat@width>\linewidth\linewidth\else\Gin@nat@width\fi}
\def\maxheight{\ifdim\Gin@nat@height>\textheight\textheight\else\Gin@nat@height\fi}
\makeatother
% Scale images if necessary, so that they will not overflow the page
% margins by default, and it is still possible to overwrite the defaults
% using explicit options in \includegraphics[width, height, ...]{}
\setkeys{Gin}{width=\maxwidth,height=\maxheight,keepaspectratio}
% Set default figure placement to htbp
\makeatletter
\def\fps@figure{htbp}
\makeatother
\setlength{\emergencystretch}{3em} % prevent overfull lines
\providecommand{\tightlist}{%
  \setlength{\itemsep}{0pt}\setlength{\parskip}{0pt}}
\setcounter{secnumdepth}{-\maxdimen} % remove section numbering
\ifLuaTeX
  \usepackage{selnolig}  % disable illegal ligatures
\fi
\IfFileExists{bookmark.sty}{\usepackage{bookmark}}{\usepackage{hyperref}}
\IfFileExists{xurl.sty}{\usepackage{xurl}}{} % add URL line breaks if available
\urlstyle{same} % disable monospaced font for URLs
\hypersetup{
  pdftitle={Bernard\_Lab6},
  pdfauthor={Kathryn Bernard},
  hidelinks,
  pdfcreator={LaTeX via pandoc}}

\title{Bernard\_Lab6}
\author{Kathryn Bernard}
\date{2022-10-15}

\begin{document}
\maketitle

\hypertarget{load-packages}{%
\subsection{Load Packages}\label{load-packages}}

\begin{Shaded}
\begin{Highlighting}[]
\FunctionTok{library}\NormalTok{(raster)}
\end{Highlighting}
\end{Shaded}

\begin{verbatim}
## Loading required package: sp
\end{verbatim}

\begin{Shaded}
\begin{Highlighting}[]
\FunctionTok{library}\NormalTok{(terra)}
\end{Highlighting}
\end{Shaded}

\begin{verbatim}
## terra 1.6.17
\end{verbatim}

\begin{Shaded}
\begin{Highlighting}[]
\FunctionTok{library}\NormalTok{(landscapemetrics)}
\FunctionTok{library}\NormalTok{(tidyverse)}
\end{Highlighting}
\end{Shaded}

\begin{verbatim}
## -- Attaching packages --------------------------------------- tidyverse 1.3.1 --
\end{verbatim}

\begin{verbatim}
## v ggplot2 3.3.6     v purrr   0.3.4
## v tibble  3.1.8     v dplyr   1.0.9
## v tidyr   1.2.0     v stringr 1.4.0
## v readr   2.1.2     v forcats 0.5.1
\end{verbatim}

\begin{verbatim}
## -- Conflicts ------------------------------------------ tidyverse_conflicts() --
## x tidyr::extract() masks terra::extract(), raster::extract()
## x dplyr::filter()  masks stats::filter()
## x dplyr::lag()     masks stats::lag()
## x dplyr::select()  masks raster::select()
\end{verbatim}

\begin{Shaded}
\begin{Highlighting}[]
\FunctionTok{library}\NormalTok{(RColorBrewer)}
\end{Highlighting}
\end{Shaded}

\hypertarget{read-in-data}{%
\subsection{Read in Data}\label{read-in-data}}

\hypertarget{using-the-raster-package}{%
\subsubsection{Using the `raster'
package}\label{using-the-raster-package}}

\begin{Shaded}
\begin{Highlighting}[]
\CommentTok{\# format of the data is a Raster* file}
\NormalTok{daneco\_2001\_orig }\OtherTok{\textless{}{-}}\NormalTok{ raster}\SpecialCharTok{::}\FunctionTok{raster}\NormalTok{(}\StringTok{"data/DaneCoNLCD2001.tif"}\NormalTok{)}
\NormalTok{daneco\_2016\_orig }\OtherTok{\textless{}{-}}\NormalTok{ raster}\SpecialCharTok{::}\FunctionTok{raster}\NormalTok{(}\StringTok{"data/DaneCoNLCD2016.tif"}\NormalTok{)}

\NormalTok{daneco\_2001\_orig}
\end{Highlighting}
\end{Shaded}

\begin{verbatim}
## class      : RasterLayer 
## dimensions : 1743, 2316, 4036788  (nrow, ncol, ncell)
## resolution : 29.98776, 29.99207  (x, y)
## extent     : 497820.5, 567272.1, 2222997, 2275273  (xmin, xmax, ymin, ymax)
## crs        : +proj=aea +lat_0=23 +lon_0=-96 +lat_1=29.5 +lat_2=45.5 +x_0=0 +y_0=0 +datum=WGS84 +units=m +no_defs 
## source     : DaneCoNLCD2001.tif 
## names      : DaneCoNLCD2001 
## values     : 11, 95  (min, max)
\end{verbatim}

\begin{Shaded}
\begin{Highlighting}[]
\NormalTok{daneco\_2016\_orig}
\end{Highlighting}
\end{Shaded}

\begin{verbatim}
## class      : RasterLayer 
## dimensions : 1743, 2316, 4036788  (nrow, ncol, ncell)
## resolution : 29.98776, 29.99207  (x, y)
## extent     : 497820.5, 567272.1, 2222997, 2275273  (xmin, xmax, ymin, ymax)
## crs        : +proj=aea +lat_0=23 +lon_0=-96 +lat_1=29.5 +lat_2=45.5 +x_0=0 +y_0=0 +datum=WGS84 +units=m +no_defs 
## source     : DaneCoNLCD2016.tif 
## names      : DaneCoNLCD2016 
## values     : 11, 95  (min, max)
\end{verbatim}

\begin{Shaded}
\begin{Highlighting}[]
\FunctionTok{plot}\NormalTok{(daneco\_2001\_orig)}
\end{Highlighting}
\end{Shaded}

\includegraphics{Bernard_Lab6_files/figure-latex/unnamed-chunk-2-1.pdf}
\includegraphics{Bernard_Lab6_files/figure-latex/unnamed-chunk-2-2.pdf}

\begin{Shaded}
\begin{Highlighting}[]
\FunctionTok{plot}\NormalTok{(daneco\_2016\_orig)}
\end{Highlighting}
\end{Shaded}

\includegraphics{Bernard_Lab6_files/figure-latex/unnamed-chunk-2-3.pdf}
\includegraphics{Bernard_Lab6_files/figure-latex/unnamed-chunk-2-4.pdf}

\hypertarget{alternatively-use-the-terra-package}{%
\subsubsection{Alternatively, use the `terra'
package}\label{alternatively-use-the-terra-package}}

\begin{Shaded}
\begin{Highlighting}[]
\CommentTok{\# format of the data is a SpatRaster* file}
\NormalTok{daneco\_2001\_terra }\OtherTok{\textless{}{-}}\NormalTok{ terra}\SpecialCharTok{::}\FunctionTok{rast}\NormalTok{(}\StringTok{"data/DaneCoNLCD2001.tif"}\NormalTok{)}
\NormalTok{daneco\_2016\_terra }\OtherTok{\textless{}{-}}\NormalTok{ terra}\SpecialCharTok{::}\FunctionTok{rast}\NormalTok{(}\StringTok{"data/DaneCoNLCD2016.tif"}\NormalTok{)}
\end{Highlighting}
\end{Shaded}

\hypertarget{reclassify-land-cover-types}{%
\subsection{Reclassify Land Cover
Types}\label{reclassify-land-cover-types}}

Using the information from the
\href{https://www.mrlc.gov/data/legends/national-land-cover-database-class-legend-and-description}{NLCD
legend}, reclassify the land cover types into 4 broader categories.

\begin{Shaded}
\begin{Highlighting}[]
\CommentTok{\# Simplify land cover types into water/snow/ice (1), developed/barren (2), vegetation (3), cropland (4)}
\NormalTok{reclass }\OtherTok{\textless{}{-}} \FunctionTok{c}\NormalTok{(}\DecValTok{11}\NormalTok{,}\DecValTok{1}\NormalTok{,}
             \DecValTok{12}\NormalTok{,}\DecValTok{1}\NormalTok{,}
             \DecValTok{21}\NormalTok{,}\DecValTok{2}\NormalTok{,}
             \DecValTok{22}\NormalTok{,}\DecValTok{2}\NormalTok{,}
             \DecValTok{23}\NormalTok{,}\DecValTok{2}\NormalTok{,}
             \DecValTok{24}\NormalTok{,}\DecValTok{2}\NormalTok{,}
             \DecValTok{31}\NormalTok{,}\DecValTok{2}\NormalTok{,}
             \DecValTok{41}\NormalTok{,}\DecValTok{3}\NormalTok{,}
             \DecValTok{42}\NormalTok{,}\DecValTok{3}\NormalTok{,}
             \DecValTok{43}\NormalTok{,}\DecValTok{3}\NormalTok{,}
             \DecValTok{51}\NormalTok{,}\DecValTok{3}\NormalTok{,}
             \DecValTok{52}\NormalTok{,}\DecValTok{3}\NormalTok{,}
             \DecValTok{71}\NormalTok{,}\DecValTok{3}\NormalTok{,}
             \DecValTok{72}\NormalTok{,}\DecValTok{3}\NormalTok{,}
             \DecValTok{73}\NormalTok{,}\DecValTok{3}\NormalTok{,}
             \DecValTok{74}\NormalTok{,}\DecValTok{3}\NormalTok{,}
             \DecValTok{81}\NormalTok{,}\DecValTok{4}\NormalTok{,}
             \DecValTok{82}\NormalTok{,}\DecValTok{4}\NormalTok{,}
             \DecValTok{90}\NormalTok{,}\DecValTok{3}\NormalTok{,}
             \DecValTok{95}\NormalTok{,}\DecValTok{3}
\NormalTok{          )}
\NormalTok{reclass\_mat }\OtherTok{\textless{}{-}} \FunctionTok{matrix}\NormalTok{(reclass,}\AttributeTok{ncol=}\DecValTok{2}\NormalTok{,}\AttributeTok{byrow=}\ConstantTok{TRUE}\NormalTok{)}

\NormalTok{daneco\_2001 }\OtherTok{\textless{}{-}}\NormalTok{ raster}\SpecialCharTok{::}\FunctionTok{reclassify}\NormalTok{(daneco\_2001\_orig,reclass\_mat)}
\NormalTok{daneco\_2016 }\OtherTok{\textless{}{-}}\NormalTok{ raster}\SpecialCharTok{::}\FunctionTok{reclassify}\NormalTok{(daneco\_2016\_orig,reclass\_mat)}
\end{Highlighting}
\end{Shaded}

\hypertarget{using-terra-instead}{%
\subsubsection{Using terra instead}\label{using-terra-instead}}

If the files were read in with the terra package, raster::reclassify
won't work (it only works for Raster* files, not SpatRasters). Terra has
its own classify function.

\begin{Shaded}
\begin{Highlighting}[]
\NormalTok{daneco\_2001\_t }\OtherTok{\textless{}{-}}\NormalTok{ terra}\SpecialCharTok{::}\FunctionTok{classify}\NormalTok{(daneco\_2001\_terra,reclass\_mat)}
\NormalTok{daneco\_2016\_t }\OtherTok{\textless{}{-}}\NormalTok{ terra}\SpecialCharTok{::}\FunctionTok{classify}\NormalTok{(daneco\_2016\_terra,reclass\_mat)}
\end{Highlighting}
\end{Shaded}

\hypertarget{landscape-metrics}{%
\subsection{Landscape Metrics}\label{landscape-metrics}}

Before calcuating metrics, check that the data is projected.

\begin{Shaded}
\begin{Highlighting}[]
\FunctionTok{check\_landscape}\NormalTok{(daneco\_2001)}
\end{Highlighting}
\end{Shaded}

\begin{verbatim}
##   layer       crs units   class n_classes OK
## 1     1 projected     m integer         4  v
\end{verbatim}

\begin{Shaded}
\begin{Highlighting}[]
\FunctionTok{check\_landscape}\NormalTok{(daneco\_2016)}
\end{Highlighting}
\end{Shaded}

\begin{verbatim}
##   layer       crs units   class n_classes OK
## 1     1 projected     m integer         4  v
\end{verbatim}

\hypertarget{calculate-the-provided-metrics}{%
\subsubsection{Calculate the provided
metrics}\label{calculate-the-provided-metrics}}

\begin{Shaded}
\begin{Highlighting}[]
\DocumentationTok{\#\# Total number of patches}
\NormalTok{tot\_patches }\OtherTok{\textless{}{-}} \FunctionTok{data.frame}\NormalTok{(}\FunctionTok{lsm\_l\_np}\NormalTok{(daneco\_2001),}\AttributeTok{year=}\StringTok{"2001"}\NormalTok{)}
\NormalTok{tot\_patches }\OtherTok{\textless{}{-}} \FunctionTok{rbind}\NormalTok{(tot\_patches,}\FunctionTok{data.frame}\NormalTok{(}\FunctionTok{lsm\_l\_np}\NormalTok{(daneco\_2016),}\AttributeTok{year=}\StringTok{"2016"}\NormalTok{))}

\DocumentationTok{\#\# Number of patches per land class}
\NormalTok{patches\_per\_class }\OtherTok{\textless{}{-}} \FunctionTok{data.frame}\NormalTok{(}\FunctionTok{lsm\_c\_np}\NormalTok{(daneco\_2001),}\AttributeTok{year=}\StringTok{"2001"}\NormalTok{)}
\NormalTok{patches\_per\_class }\OtherTok{\textless{}{-}} \FunctionTok{rbind}\NormalTok{(patches\_per\_class,}\FunctionTok{data.frame}\NormalTok{(}\FunctionTok{lsm\_c\_np}\NormalTok{(daneco\_2016),}\AttributeTok{year=}\StringTok{"2016"}\NormalTok{))}

\DocumentationTok{\#\# Total edge}
\CommentTok{\# Units = meters}
\NormalTok{tot\_edge }\OtherTok{\textless{}{-}} \FunctionTok{data.frame}\NormalTok{(}\FunctionTok{lsm\_l\_te}\NormalTok{(daneco\_2001),}\AttributeTok{year=}\StringTok{"2001"}\NormalTok{)}
\NormalTok{tot\_edge }\OtherTok{\textless{}{-}} \FunctionTok{rbind}\NormalTok{(tot\_edge,}\FunctionTok{data.frame}\NormalTok{(}\FunctionTok{lsm\_l\_te}\NormalTok{(daneco\_2016),}\AttributeTok{year=}\StringTok{"2016"}\NormalTok{))}

\DocumentationTok{\#\# Patch areas}
\CommentTok{\# units = hectares}
\NormalTok{patch\_area }\OtherTok{\textless{}{-}} \FunctionTok{data.frame}\NormalTok{(}\FunctionTok{lsm\_p\_area}\NormalTok{(daneco\_2001),}\AttributeTok{year=}\StringTok{"2001"}\NormalTok{)}
\NormalTok{patch\_area }\OtherTok{\textless{}{-}} \FunctionTok{rbind}\NormalTok{(patch\_area,}\FunctionTok{data.frame}\NormalTok{(}\FunctionTok{lsm\_p\_area}\NormalTok{(daneco\_2016),}\AttributeTok{year=}\StringTok{"2016"}\NormalTok{))}
\NormalTok{mean\_patch\_area }\OtherTok{\textless{}{-}}\NormalTok{ patch\_area }\SpecialCharTok{\%\textgreater{}\%} \FunctionTok{group\_by}\NormalTok{(year) }\SpecialCharTok{\%\textgreater{}\%} \FunctionTok{summarise}\NormalTok{(}\AttributeTok{avg=}\FunctionTok{mean}\NormalTok{(value))}

\DocumentationTok{\#\# Proportion of like adjacency}
\CommentTok{\# PLADJ is an ’Aggregation metric’. It calculates the frequency how often patches of different classes}
\CommentTok{\# i (focal class) and k are next to each other, and following is a measure of class aggregation. The}
\CommentTok{\# adjacencies are counted using the double{-}count method.}

\NormalTok{prop\_like\_adj }\OtherTok{\textless{}{-}} \FunctionTok{data.frame}\NormalTok{(}\FunctionTok{lsm\_l\_pladj}\NormalTok{(daneco\_2001),}\AttributeTok{year=}\StringTok{"2001"}\NormalTok{)}
\NormalTok{prop\_like\_adj }\OtherTok{\textless{}{-}} \FunctionTok{rbind}\NormalTok{(prop\_like\_adj,}\FunctionTok{data.frame}\NormalTok{(}\FunctionTok{lsm\_l\_pladj}\NormalTok{(daneco\_2016),}\AttributeTok{year=}\StringTok{"2016"}\NormalTok{))}
\end{Highlighting}
\end{Shaded}

\hypertarget{calculate-additional-metrics}{%
\subsubsection{Calculate additional
metrics}\label{calculate-additional-metrics}}

\begin{Shaded}
\begin{Highlighting}[]
\DocumentationTok{\#\# Patch Cohesion Index}
\CommentTok{\# COHESION is an ’Aggregation metric’. It characterises the connectedness of patches belonging to}
\CommentTok{\# class i. It can be used to asses if patches of the same class are located aggregated or rather isolated}
\CommentTok{\# and thereby COHESION gives information about the configuration of the landscape.}
\CommentTok{\# Range from 0 to 100 (0 = isolated)}
\CommentTok{\# Unit is \%s}

\NormalTok{class\_cohesion }\OtherTok{\textless{}{-}} \FunctionTok{data.frame}\NormalTok{(}\FunctionTok{lsm\_c\_cohesion}\NormalTok{(daneco\_2001, }\AttributeTok{directions =} \DecValTok{4}\NormalTok{),}\AttributeTok{year=}\StringTok{"2001"}\NormalTok{) }\CommentTok{\# rook\textquotesingle{}s case (4 nearest neighbors)}
\NormalTok{class\_cohesion }\OtherTok{\textless{}{-}} \FunctionTok{rbind}\NormalTok{(class\_cohesion,}\FunctionTok{data.frame}\NormalTok{(}\FunctionTok{lsm\_c\_cohesion}\NormalTok{(daneco\_2016, }\AttributeTok{directions =} \DecValTok{4}\NormalTok{),}\AttributeTok{year=}\StringTok{"2016"}\NormalTok{))}

\DocumentationTok{\#\# Simpson\textquotesingle{}s Evenness Index}
\CommentTok{\# SIEI is a \textquotesingle{}Diversity metric\textquotesingle{}. The metric is widely used in biodiversity and ecology.}
\CommentTok{\# It is the ratio between the actual Simpson\textquotesingle{}s diversity index and the theoretical maximum}
\CommentTok{\# Simpson\textquotesingle{}s diversity index.}
\CommentTok{\# Ranges from 0 to 1}
\CommentTok{\# Equals SIEI = 0 when only one patch is present and approaches SIEI = 1 when the number}
\CommentTok{\# of class types increases while the proportions are equally distributed}

\NormalTok{siei }\OtherTok{\textless{}{-}} \FunctionTok{data.frame}\NormalTok{(}\FunctionTok{lsm\_l\_siei}\NormalTok{(daneco\_2001, }\AttributeTok{directions =} \DecValTok{4}\NormalTok{),}\AttributeTok{year=}\StringTok{"2001"}\NormalTok{)}
\NormalTok{siei }\OtherTok{\textless{}{-}} \FunctionTok{rbind}\NormalTok{(siei,}\FunctionTok{data.frame}\NormalTok{(}\FunctionTok{lsm\_l\_siei}\NormalTok{(daneco\_2016, }\AttributeTok{directions =} \DecValTok{4}\NormalTok{),}\AttributeTok{year=}\StringTok{"2016"}\NormalTok{))}

\DocumentationTok{\#\# Shannon\textquotesingle{}s Evenness Index}
\CommentTok{\# SHEI is a \textquotesingle{}Diversity metric\textquotesingle{}. It is the ratio between the actual}
\CommentTok{\# Shannon\textquotesingle{}s diversity index and and the theoretical maximum of the Shannon diversity index.}
\CommentTok{\# It can be understood as a measure of dominance.}
\CommentTok{\# Ranges from 0 to 1}
\CommentTok{\# Equals SHEI = 0 when only one patch present and equals SHEI = 1 when the proportion of}
\CommentTok{\# classes is completely equally distributed}

\NormalTok{shei }\OtherTok{\textless{}{-}} \FunctionTok{data.frame}\NormalTok{(}\FunctionTok{lsm\_l\_shei}\NormalTok{(daneco\_2001),}\AttributeTok{year=}\StringTok{"2001"}\NormalTok{)}
\NormalTok{shei }\OtherTok{\textless{}{-}} \FunctionTok{rbind}\NormalTok{(shei,}\FunctionTok{data.frame}\NormalTok{(}\FunctionTok{lsm\_l\_shei}\NormalTok{(daneco\_2016),}\AttributeTok{year=}\StringTok{"2016"}\NormalTok{))}
\end{Highlighting}
\end{Shaded}

\hypertarget{visualize-the-metrics}{%
\subsection{Visualize the Metrics}\label{visualize-the-metrics}}

\hypertarget{set-up-color-palette-and-theme}{%
\subsubsection{Set up color palette and
theme}\label{set-up-color-palette-and-theme}}

\begin{Shaded}
\begin{Highlighting}[]
\CommentTok{\# Color palette}
\NormalTok{redylbu }\OtherTok{\textless{}{-}} \FunctionTok{brewer.pal}\NormalTok{(}\AttributeTok{name=}\StringTok{"RdYlBu"}\NormalTok{,}\AttributeTok{n=}\DecValTok{9}\NormalTok{)}
\NormalTok{colors }\OtherTok{\textless{}{-}} \FunctionTok{c}\NormalTok{(redylbu[}\DecValTok{2}\NormalTok{], redylbu[}\DecValTok{4}\NormalTok{])}
\NormalTok{outline }\OtherTok{\textless{}{-}}\NormalTok{ redylbu[}\DecValTok{1}\NormalTok{]}

\CommentTok{\# Custom theme}
\NormalTok{my\_theme }\OtherTok{\textless{}{-}} \FunctionTok{theme\_minimal}\NormalTok{() }\SpecialCharTok{+}
  \FunctionTok{theme}\NormalTok{(}\AttributeTok{plot.title =} \FunctionTok{element\_text}\NormalTok{(}\AttributeTok{hjust =} \FloatTok{0.5}\NormalTok{),}
        \AttributeTok{plot.subtitle=}\FunctionTok{element\_text}\NormalTok{(}\AttributeTok{face=}\StringTok{"italic"}\NormalTok{, }\AttributeTok{color=}\StringTok{"gray"}\NormalTok{))}
\end{Highlighting}
\end{Shaded}

\hypertarget{compare-2001-and-2016-measurements}{%
\subsubsection{Compare 2001 and 2016
measurements}\label{compare-2001-and-2016-measurements}}

\begin{Shaded}
\begin{Highlighting}[]
\CommentTok{\# Total number of patches}
\FunctionTok{ggplot}\NormalTok{(tot\_patches, }\FunctionTok{aes}\NormalTok{(}\AttributeTok{x=}\NormalTok{year,}\AttributeTok{y=}\NormalTok{value,}\AttributeTok{fill=}\NormalTok{year)) }\SpecialCharTok{+} 
  \FunctionTok{geom\_bar}\NormalTok{(}\AttributeTok{stat=}\StringTok{"identity"}\NormalTok{,}\AttributeTok{color=}\NormalTok{outline) }\SpecialCharTok{+}
  \FunctionTok{geom\_label}\NormalTok{(}\AttributeTok{label=}\NormalTok{tot\_patches}\SpecialCharTok{$}\NormalTok{value,}\AttributeTok{fill=}\StringTok{"white"}\NormalTok{) }\SpecialCharTok{+}
  \FunctionTok{scale\_fill\_manual}\NormalTok{(}\AttributeTok{values=}\NormalTok{colors) }\SpecialCharTok{+}
  \FunctionTok{labs}\NormalTok{(}\AttributeTok{x=}\StringTok{"Year"}\NormalTok{,}\AttributeTok{y=}\StringTok{"Number of Patches"}\NormalTok{,}\AttributeTok{title=}\StringTok{"Total Number of Patches in Dane County"}\NormalTok{,}
       \AttributeTok{subtitle=}\StringTok{"Average patch area in 2001: 21.8 hectares}\SpecialCharTok{\textbackslash{}n}\StringTok{Average patch area in 2016: 20.6 hectares"}\NormalTok{) }\SpecialCharTok{+}
  \FunctionTok{guides}\NormalTok{(}\AttributeTok{fill=}\StringTok{"none"}\NormalTok{) }\SpecialCharTok{+}
\NormalTok{  my\_theme}
\end{Highlighting}
\end{Shaded}

\includegraphics{Bernard_Lab6_files/figure-latex/unnamed-chunk-10-1.pdf}

\begin{Shaded}
\begin{Highlighting}[]
\CommentTok{\# Number of patches per land class}
\FunctionTok{ggplot}\NormalTok{(patches\_per\_class, }\FunctionTok{aes}\NormalTok{(}\AttributeTok{x=}\FunctionTok{factor}\NormalTok{(class),}\AttributeTok{y=}\NormalTok{value,}\AttributeTok{fill=}\NormalTok{year)) }\SpecialCharTok{+} 
  \FunctionTok{geom\_bar}\NormalTok{(}\AttributeTok{stat=}\StringTok{"identity"}\NormalTok{,}\AttributeTok{position=}\StringTok{"dodge"}\NormalTok{,}\AttributeTok{color=}\NormalTok{outline) }\SpecialCharTok{+}
  \CommentTok{\#geom\_label(label=patches\_per\_class$value,position=position\_dodge(0.9)) +}
  \FunctionTok{scale\_x\_discrete}\NormalTok{(}\AttributeTok{labels=}\FunctionTok{c}\NormalTok{(}\StringTok{"1"}\OtherTok{=}\StringTok{"Water"}\NormalTok{,}\StringTok{"2"}\OtherTok{=}\StringTok{"Developed"}\NormalTok{,}\StringTok{"3"}\OtherTok{=}\StringTok{"Vegetation"}\NormalTok{,}\StringTok{"4"}\OtherTok{=}\StringTok{"Cropland"}\NormalTok{)) }\SpecialCharTok{+}
  \FunctionTok{scale\_fill\_manual}\NormalTok{(}\AttributeTok{values=}\NormalTok{colors,}\AttributeTok{name=}\StringTok{"Year"}\NormalTok{) }\SpecialCharTok{+}
  \FunctionTok{labs}\NormalTok{(}\AttributeTok{x=}\StringTok{"Land Cover Class"}\NormalTok{,}\AttributeTok{y=}\StringTok{"Number of Patches"}\NormalTok{,}\AttributeTok{title=}\StringTok{"Number of Patches Per Land Cover Class in Dane County"}\NormalTok{) }\SpecialCharTok{+}
\NormalTok{  my\_theme}
\end{Highlighting}
\end{Shaded}

\includegraphics{Bernard_Lab6_files/figure-latex/unnamed-chunk-10-2.pdf}

\begin{Shaded}
\begin{Highlighting}[]
\CommentTok{\# Total edge}
\CommentTok{\#tot\_edge \%\textgreater{}\% mutate(value\_km \textless{}{-} value/1000) \%\textgreater{}\% }
\FunctionTok{ggplot}\NormalTok{(tot\_edge,}\FunctionTok{aes}\NormalTok{(}\AttributeTok{x=}\NormalTok{year,}\AttributeTok{y=}\NormalTok{value}\SpecialCharTok{/}\DecValTok{1000}\NormalTok{,}\AttributeTok{fill=}\NormalTok{year)) }\SpecialCharTok{+} 
  \FunctionTok{geom\_bar}\NormalTok{(}\AttributeTok{stat=}\StringTok{"identity"}\NormalTok{,}\AttributeTok{color=}\NormalTok{outline) }\SpecialCharTok{+}
  \FunctionTok{geom\_label}\NormalTok{(}\AttributeTok{label=}\FunctionTok{round}\NormalTok{(tot\_edge}\SpecialCharTok{$}\NormalTok{value}\SpecialCharTok{/}\DecValTok{1000}\NormalTok{,}\DecValTok{2}\NormalTok{),}\AttributeTok{fill=}\StringTok{"white"}\NormalTok{) }\SpecialCharTok{+}
  \FunctionTok{scale\_fill\_manual}\NormalTok{(}\AttributeTok{values=}\NormalTok{colors) }\SpecialCharTok{+}
  \FunctionTok{labs}\NormalTok{(}\AttributeTok{x=}\StringTok{"Year"}\NormalTok{,}\AttributeTok{y=}\StringTok{"Edge Length (km)"}\NormalTok{,}\AttributeTok{title=}\StringTok{"Total Edge Length of Patches in Dane County"}\NormalTok{) }\SpecialCharTok{+}
  \FunctionTok{guides}\NormalTok{(}\AttributeTok{fill=}\StringTok{"none"}\NormalTok{) }\SpecialCharTok{+}
\NormalTok{  my\_theme}
\end{Highlighting}
\end{Shaded}

\includegraphics{Bernard_Lab6_files/figure-latex/unnamed-chunk-10-3.pdf}

\begin{Shaded}
\begin{Highlighting}[]
\CommentTok{\# Proportion of like adjacency}
\FunctionTok{ggplot}\NormalTok{(prop\_like\_adj,}\FunctionTok{aes}\NormalTok{(}\AttributeTok{x=}\NormalTok{year,}\AttributeTok{y=}\NormalTok{value,}\AttributeTok{fill=}\NormalTok{year)) }\SpecialCharTok{+} 
  \FunctionTok{geom\_bar}\NormalTok{(}\AttributeTok{stat=}\StringTok{"identity"}\NormalTok{,}\AttributeTok{color=}\NormalTok{outline) }\SpecialCharTok{+}
  \FunctionTok{geom\_label}\NormalTok{(}\AttributeTok{label=}\FunctionTok{round}\NormalTok{(prop\_like\_adj}\SpecialCharTok{$}\NormalTok{value,}\DecValTok{2}\NormalTok{),}\AttributeTok{fill=}\StringTok{"white"}\NormalTok{) }\SpecialCharTok{+}
  \FunctionTok{scale\_fill\_manual}\NormalTok{(}\AttributeTok{values=}\NormalTok{colors) }\SpecialCharTok{+}
  \FunctionTok{labs}\NormalTok{(}\AttributeTok{x=}\StringTok{"Year"}\NormalTok{,}\AttributeTok{y=}\StringTok{"Proportion of Like Adjacency (\%)"}\NormalTok{,}\AttributeTok{title=}\StringTok{"Proportion of Like Adjacency for Patches in Dane County"}\NormalTok{) }\SpecialCharTok{+}
  \FunctionTok{guides}\NormalTok{(}\AttributeTok{fill=}\StringTok{"none"}\NormalTok{) }\SpecialCharTok{+}
\NormalTok{  my\_theme}
\end{Highlighting}
\end{Shaded}

\includegraphics{Bernard_Lab6_files/figure-latex/unnamed-chunk-10-4.pdf}

\begin{Shaded}
\begin{Highlighting}[]
\CommentTok{\# Patch cohesion index}
\FunctionTok{ggplot}\NormalTok{(class\_cohesion,}\FunctionTok{aes}\NormalTok{(}\AttributeTok{x=}\FunctionTok{factor}\NormalTok{(class),}\AttributeTok{y=}\NormalTok{value,}\AttributeTok{fill=}\NormalTok{year)) }\SpecialCharTok{+} 
  \FunctionTok{geom\_bar}\NormalTok{(}\AttributeTok{stat=}\StringTok{"identity"}\NormalTok{,}\AttributeTok{position=}\StringTok{"dodge"}\NormalTok{,}\AttributeTok{color=}\NormalTok{outline) }\SpecialCharTok{+}
  \CommentTok{\#geom\_label(label=round(class\_cohesion$value,2),fill="white") +}
  \FunctionTok{scale\_x\_discrete}\NormalTok{(}\AttributeTok{labels=}\FunctionTok{c}\NormalTok{(}\StringTok{"1"}\OtherTok{=}\StringTok{"Water"}\NormalTok{,}\StringTok{"2"}\OtherTok{=}\StringTok{"Developed"}\NormalTok{,}\StringTok{"3"}\OtherTok{=}\StringTok{"Vegetation"}\NormalTok{,}\StringTok{"4"}\OtherTok{=}\StringTok{"Cropland"}\NormalTok{)) }\SpecialCharTok{+}
  \FunctionTok{scale\_fill\_manual}\NormalTok{(}\AttributeTok{values=}\NormalTok{colors,}\AttributeTok{name=}\StringTok{"Year"}\NormalTok{) }\SpecialCharTok{+}
  \FunctionTok{labs}\NormalTok{(}\AttributeTok{x=}\StringTok{"Land Cover Class"}\NormalTok{,}\AttributeTok{y=}\StringTok{"Class Cohesion Index (\%)"}\NormalTok{,}\AttributeTok{title=}\StringTok{"Class Cohesion Indices for Patches in Dane County"}\NormalTok{) }\SpecialCharTok{+}
  \FunctionTok{guides}\NormalTok{(}\AttributeTok{fill=}\StringTok{"none"}\NormalTok{) }\SpecialCharTok{+}
\NormalTok{  my\_theme}
\end{Highlighting}
\end{Shaded}

\includegraphics{Bernard_Lab6_files/figure-latex/unnamed-chunk-10-5.pdf}

\begin{Shaded}
\begin{Highlighting}[]
\CommentTok{\# SEIE}
\FunctionTok{ggplot}\NormalTok{(siei,}\FunctionTok{aes}\NormalTok{(}\AttributeTok{x=}\NormalTok{year,}\AttributeTok{y=}\NormalTok{value,}\AttributeTok{fill=}\NormalTok{year)) }\SpecialCharTok{+} 
  \FunctionTok{geom\_bar}\NormalTok{(}\AttributeTok{stat=}\StringTok{"identity"}\NormalTok{,}\AttributeTok{color=}\NormalTok{outline) }\SpecialCharTok{+}
  \FunctionTok{geom\_label}\NormalTok{(}\AttributeTok{label=}\FunctionTok{round}\NormalTok{(siei}\SpecialCharTok{$}\NormalTok{value,}\DecValTok{2}\NormalTok{),}\AttributeTok{fill=}\StringTok{"white"}\NormalTok{) }\SpecialCharTok{+}
  \FunctionTok{scale\_fill\_manual}\NormalTok{(}\AttributeTok{values=}\NormalTok{colors) }\SpecialCharTok{+}
  \FunctionTok{labs}\NormalTok{(}\AttributeTok{x=}\StringTok{"Year"}\NormalTok{,}\AttributeTok{y=}\StringTok{"Simpson\textquotesingle{}s Evenness Index"}\NormalTok{,}\AttributeTok{title=}\StringTok{"Simpson\textquotesingle{}s Evenness Index for Dane County"}\NormalTok{) }\SpecialCharTok{+}
  \FunctionTok{guides}\NormalTok{(}\AttributeTok{fill=}\StringTok{"none"}\NormalTok{) }\SpecialCharTok{+}
\NormalTok{  my\_theme}
\end{Highlighting}
\end{Shaded}

\includegraphics{Bernard_Lab6_files/figure-latex/unnamed-chunk-10-6.pdf}

\begin{Shaded}
\begin{Highlighting}[]
\CommentTok{\# SHEI}
\FunctionTok{ggplot}\NormalTok{(shei,}\FunctionTok{aes}\NormalTok{(}\AttributeTok{x=}\NormalTok{year,}\AttributeTok{y=}\NormalTok{value,}\AttributeTok{fill=}\NormalTok{year)) }\SpecialCharTok{+} 
  \FunctionTok{geom\_bar}\NormalTok{(}\AttributeTok{stat=}\StringTok{"identity"}\NormalTok{,}\AttributeTok{color=}\NormalTok{outline) }\SpecialCharTok{+}
  \FunctionTok{geom\_label}\NormalTok{(}\AttributeTok{label=}\FunctionTok{round}\NormalTok{(shei}\SpecialCharTok{$}\NormalTok{value,}\DecValTok{2}\NormalTok{),}\AttributeTok{fill=}\StringTok{"white"}\NormalTok{) }\SpecialCharTok{+}
  \FunctionTok{scale\_fill\_manual}\NormalTok{(}\AttributeTok{values=}\NormalTok{colors) }\SpecialCharTok{+}
  \FunctionTok{labs}\NormalTok{(}\AttributeTok{x=}\StringTok{"Year"}\NormalTok{,}\AttributeTok{y=}\StringTok{"Shannon\textquotesingle{}s Evenness Index"}\NormalTok{,}\AttributeTok{title=}\StringTok{"Shannon\textquotesingle{}s Evenness Index for Dane County"}\NormalTok{) }\SpecialCharTok{+}
  \FunctionTok{guides}\NormalTok{(}\AttributeTok{fill=}\StringTok{"none"}\NormalTok{) }\SpecialCharTok{+}
\NormalTok{  my\_theme}
\end{Highlighting}
\end{Shaded}

\includegraphics{Bernard_Lab6_files/figure-latex/unnamed-chunk-10-7.pdf}

\end{document}
